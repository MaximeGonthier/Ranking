\documentclass[a4paper,11pt]{article}
\usepackage[utf8]{inputenc}
\usepackage[T1]{fontenc}
\usepackage[french]{babel}


\title{PROJET\\ UE Méthodes de ranking et recommandations\\ 
		Sujet 5 : Simulation d'un Google Bombing}
\author{Maxime Gonthier - Laureline Martin}
\begin{document}

\pagenumbering{gobble}\clearpage
	\maketitle

\newpage
\tableofcontents

\newpage
\section{TESTS INITIAUX (ET HYPOTHESES?)}
	\subsection{Pagerank sur Stanford.txt (sans modification) :}
		281903 pages\\
		2312497 liens\\
		132 itérations\\
		27.627466 secondes\\
		\\
		On choisi d'attaquer ces différentes pages à pertinences différentes :\\
		Pertinence forte : Page 280545 9.96199e-05\\
		Pertinence moyenne : Page 281466 7.53954e-06\\
		Pertinence faible : Page 281574 6.05222e-07\\
		\\
	\subsection{Pagerank sur Stanford.txt (ajout de 10 sommets seuls) :}
		281913 pages\\
		2312507 liens\\
		132 itérations\\
		27.075775 secondes / 27.082188 secondes / 28.454464 secondes\\
		\\
		Cible à pertinence forte : 1.07873e-04\\
		Cible à pertinence moyenne : 1.26718e-05\\
		Cible à pertinence faible : 5.73756e-06\\
		\\
	\subsection{Pagerank sur Stanford.txt (ajout d'un anneau à 10 sommets) :}
		281913 pages\\
		2312508 liens\\
		132 itérations\\
		29.058716 secondes / 27.939356 secondes / 30.209354 secondes\\
		\\
		Cible à pertinence forte : 1.02068e-04\\
		Cible à pertinence moyenne : 9.06326e-06\\
		Cible à pertinence faible : 2.12916e-06\\
		\\
	\subsection{Pagerank sur Stanford.txt (ajout d'un graphe complet à 10 sommets) :}
		281913 pages\\
		2312588 liens\\
		132 itérations\\
		32.530811 secondes / 30.358072 secondes / 28.396551 secondes\\
		\\
		Cible à pertinence forte : 1.00141e-04\\	
		Cible à pertinence moyenne : 7.86544e-06\\
		Cible à pertinence faible : 9.31393e-07\\
		\\
	\subsection{Pagerank sur Stanford.txt (ajout d'un arbre binaire à 10 sommets) :}
		281913 pages\\
		2312507 liens\\
		132 itérations\\
		30.578079 secondes / 31.169865 secondes / 28.670444 secondes\\
		\\
		Cible à pertinence forte : 1.05753e-04\\
		Cible à pertinence moyenne : 1.13539e-05\\
		Cible à pertinence faible : 4.41974e-06\\
		\\
\section{ANALYSES}
	A chaque test, nous ajoutons 10 sommets au graphe (sous différentes strucutes). 10 sommets correspondent à une augmentation du graphe de 0,0035\%.\\

	\subsection{Analyse des modifications de pertinences par l'ajout de 10 sommets seuls :}
		Cible à pertinence forte :
		\begin{itemize} 	
			\item Ajout d'une pertinence de 8,2531e-06
			\item Augmentation de la pertinence de 8.2846e-10 \%.
		\end{itemize}
		
		Cible à pertinence moyenne :
		\begin{itemize} 	
			\item Ajout d'une pertinence de 5,13226e-06
			\item Augmentation de la pertinence de 6.801713e-11 \%.
		\end{itemize}
		
		Cible à pertinence faible :
		\begin{itemize} 	
			\item Ajout d'une pertinence de 5.132338 e-06
			\item Augmentation de la pertinence de 8.48009 e-12 \%.
		\end{itemize}

	\subsection{Analyse des modifications de pertinences par l'ajout d'un anneau à 10 sommets seuls :}
	Cible à pertinence forte :
		\begin{itemize} 	
			\item Ajout d'une pertinence de 
			\item Augmentation de la pertinence de  \%.
		\end{itemize}
		
		Cible à pertinence moyenne :
		\begin{itemize} 	
			\item Ajout d'une pertinence de 
			\item Augmentation de la pertinence de  \%.
		\end{itemize}
		
		Cible à pertinence faible :
		\begin{itemize} 	
			\item Ajout d'une pertinence de 
			\item Augmentation de la pertinence de  \%.
		\end{itemize}




















\end{document}